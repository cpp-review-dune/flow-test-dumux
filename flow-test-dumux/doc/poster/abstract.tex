\documentclass[a4paper,abstract=true]{scrartcl}
\usepackage[colorlinks]{hyperref}
\usepackage{filecontents}

\begin{filecontents}{\jobname.bib}
@online{bioarchlinux_2022,
	Title = {BioArchLinux: bioinformatics community with Arch Linux \texttt{https://doi.org/10.7490/f1000research.1119039.1}},
	Author = {Zhang G, Hu Y, Drobot V},
	Month = {July},
	url = {https://doi.org/10.7490/f1000research.1119039.1},
	Year = {2022}
}
@online{gitpod_2022,
	Title = {Gitpod an open-source Kubernetes \texttt{https://github.com/gitpod-io/gitpod}},
	Author = {Manuel de Brito Fontes, Christian Weichel},
	Month = {July},
	url = {https://github.com/gitpod-io/gitpod},
	Year = {2018}
}
@online{arch4edu2019,
	Title = {Archlinux and {A}rchlinux{ARM} {R}epository for {E}ducation \texttt{https://github.com/arch4edu/arch4edu}},
	Author = {Jingbei Li, Carlos Aznarán},
	Month = {January},
	url = {https://github.com/arch4edu/arch4edu},
	Year = {2019}
}
\end{filecontents}
\newcommand{\MVAt}{{\usefont{U}{mvs}{m}{n}\symbol{`@}}}
\bibliographystyle{plain}
\date{}


\title{Virtualization of scientific software based on Arch Linux in GitPod}
\author{Carlos A. Aznarán Laos
\thanks{
  Universidad Nacional de Ingeniería,
\texttt{caznaranl\MVAt uni.pe}}
    \and John J. Leal Gomez
    \thanks{
    Universidad Nacional de Colombia,
		\texttt{jlealgom\MVAt unal.edu.co}}
}
\pagestyle{empty}
\begin{document}
\maketitle

\begin{abstract}
	We developed an open source repository hosted on GitHub to use
	scientific software based on the \href{https://archlinux.org}{Arch Linux distribution},
	with an up-to-date software ecosystem that includes
	\href{https://dune-project.org/doc/gettingstarted}{DUNE python bindings},
	\href{https://dumux.org}{DuMu\textsuperscript{x}},
	\href{https://fenicsproject.org}{FEniCS},
	\href{https://www.dealii.org}{deal.II},
	\href{https://gmsh.info}{Gmsh},
	\href{https://precice.org/adapters-overview.html}{preCICE adapters},
	among others.
	Unlike other projects such as
	\href{https://github.com/BioArchLinux}{BioArchLinux}~\cite{bioarchlinux_2022}
	or \href{https://github.com/arch4edu}{Arch Linux for education}~\cite{arch4edu2019},
	we include some tutorials on GitHub Classroom to allow the practice
	to any newcomer.
	Automated deployed and available for free use allowing virtualization inside
	GitPod~\cite{gitpod_2022}.
\end{abstract}

\bibliography{\jobname}

\end{document}