\section{DUNE for Multi-\{Phase, Component, Scale, Physics, \ldots\} flow and transport in porous media}
\subsection{Introduction}

\begin{frame}
	\frametitle{\secname}
	\framesubtitle{\subsecname}

	\begin{alertblock}{DUNE for Multi-\{Phase, Component, Scale, Physics, \ldots\} flow and transport in porous media (DuMu\textsuperscript{x})}
		\note{
			Título: ``Una introducción a la caja de herramientas DUNE en C++/Python
			para la solución de modelos matemáticos''.
			Se hará una breve presentación de  la caja de herramientas modular Dune Numerics, biblioteca modular desarrollada en la  Universidad de Heildeberg en C++ y Python, para resolver ecuaciones
			diferenciales parciales utilizando métodos basados en mallas, por ejemplo diferencias finitas, elementos finitos o volúmenes finitos.

			Es un software  de código abierto bajo la licencia GNU General Public Licence 2, con binarios disponibles para las distribuciones linux Debian, Ubuntu y openSUSE; y
			los scripts de compilación en macOS, FreeBSD, Arch Linux.

			Se mostrará la estructura general, algunos proyectos basados en DUNE y algunas simulaciones de modelos matemáticos que incluyen éste tipo de ecuaciones y sus respectivas soluciones, así como una implementación breve de Dune Numerics.
		}
		\begin{itemize}
			\item

			      Software de código abierto bajo la licencia GNU Lesser General Public License 3~\lgpllicense{}.

			% \item

			%       Disponible en
			%       \href{https://github.com/dune-copasi/homebrew-tap}{macOS}, \href{https://packages.debian.org/search?suite=sid&section=all&arch=any&searchon=sourcenames&keywords=dune-}{Debian}~\debian{},
			%       \href{https://launchpad.net/~opm/+archive/ubuntu/ppa}{Ubuntu}~\ubuntu{},
			%       \href{https://build.opensuse.org/search?search_text=dune-&search_for=2&name=1&attrib_type_id=}{openSUSE}~\opensuse{},
			%       \href{https://aur.archlinux.org/packages/?O=0&SeB=n&K=dune-&outdated=&SB=n&SO=a&PP=50&do_Search=Ir}{Arch Linux}~\archlinux{} y \href{https://www.freshports.org/search.php?stype=name&method=match&query=dune-&num=20&orderby=category&orderbyupdown=asc&search=Search&format=html&branch=head}{FreeBSD}~\freebsd{}.

			% \item

			%       Conjunto de bibliotecas C++ con enlaces a \href{https://pypi.org/search/?q=dune-}{Python}.

			%       \note{
			% 	      Desarrollado con CMake, escrito en C++ con enlaces Python a través de pybind11.

			%       }

			% \item

			%       Utilizado en la resolución de ecuaciones diferenciales parciales e implementación de métodos basados en mallas, por ejemplo diferencias finitas, elementos finitos o volúmenes finitos.
		\end{itemize}
	\end{alertblock}

	% \begin{figure}[ht!]
	% 	\centering
	% 	\includegraphics[height=4.2cm]{dunedesign}
	% 	\caption{Tomado de \url{https://dune-project.org}.}
	% \end{figure}

\end{frame}

% documentación https://git.iws.uni-stuttgart.de/dumux-repositories/dumux-course
% página principal
% lenguajes