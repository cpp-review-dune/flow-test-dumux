\documentclass{beamer}

\newcommand{\MVAt}{{\usefont{U}{mvs}{m}{n}\symbol{`@}}}

\title{Richards model for simulations of water infiltration in agricultural soil using DuMu\textsuperscript{x}}
\author{John J. Leal Gomez
	\thanks{
		Universidad Nacional de Colombia,
		\texttt{jlealgom\MVAt unal.edu.co}}
    \and Guillermo A. Martínez Girón
    \thanks{
    Universidad Nacional de Colombia,
		\texttt{jlealgom\MVAt unal.edu.co}}
    \and Carlos A. Aznarán Laos
    \thanks{
      Universidad Nacional de Ingeniería,
		\texttt{caznaranl\MVAt uni.pe}}
}

\begin{document}

% 1. Presentación del problema de infiltración de agua
% 1.1 Ventajas de la simulación % Arley ayuda, sin tantas ecuaciones.
% 2. La ecuación de Richards
% 2.1. Deducción de la ecuación de Richards (1D o 2D) (Mencionar Darcy)
% 2.1.1 Solución analítica y sus curvas solución con gráficas.
% 2.2.1 Van Genutchen Brookes-Corey model (opcional, pero recomendable)
% 3. El suelo y las características
% 3.1 Parámetros del suelo (triángulo estructural) Guillermo
% 3.2 Perfil del suelo, foto de Laboratorio. % Ayuda de Arley, Guillermo.
% 4. Resultados de la simulación % enlace a la página con el vídeo, un archivo tar.gz
% 4.1. DuMuX (1 hoja)
% 4.1.1 Modelos TFFA, black-box (posiblemente RichardsFOAM)
% 
% Bibliograf'ia

\begin{frame}
	\maketitle
\end{frame}

\begin{frame}

\end{frame}

\begin{frame}

\end{frame}


\begin{frame}

\end{frame}

\end{document}