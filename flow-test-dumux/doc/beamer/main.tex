\documentclass[
	% spanish,
	9pt,
	xcolor=table,
	handout,
	aspectratio=169,
	c,
	%ignorenonframetext
]{beamer}

% \usepackage[spanish]{babel}
\usepackage[
	cursointroductorio,
	fullpagenumbering
]{dunestyle-beamer}
\usepackage{caption}
%\usepackage{svg}
%\graphicspath{{./images}}
\usepackage{booktabs}
\usepackage{longtable}
\usepackage{multicol}
%\setbeameroption{show notes on second screen=right}
\usepackage[
	backend=biber,
	style=numeric,
	defernumbers=true,
	sorting=ynt,
	maxbibnames=4,
	maxcitenames=4
]{biblatex}
\addbibresource{dumux-conference-references.bib}

\title{
	Richards model for simulations of water infiltration in
	agricultural soil using DuMu\textsuperscript{x}
}
\subtitle{
	Parte I:
	\lstinline{bash}
}

\author{}
\institute[]
{
	\noindent
	Practice the examples on GitPod

	\href{https://gitpod.io\#/https://github.com/cpp-review-dune/flow-test-dumux}{
		\includegraphics[width=4.5cm]{open-in-gitpod}
	}
}

\setbeamertemplate{bibliography item}{%
	\ifboolexpr{ test {\ifentrytype{book}} or test {\ifentrytype{mvbook}}
		or test {\ifentrytype{collection}} or test {\ifentrytype{mvcollection}}
		or test {\ifentrytype{reference}} or test {\ifentrytype{mvreference}} }
	{\setbeamertemplate{bibliography item}[book]}
	{\ifentrytype{online}
		{\setbeamertemplate{bibliography item}[online]}
		{\setbeamertemplate{bibliography item}[article]}}%
	\usebeamertemplate{bibliography item}}

\defbibenvironment{bibliography}
{\list{}
	{\settowidth{\labelwidth}{\usebeamertemplate{bibliography item}}%
		\setlength{\leftmargin}{\labelwidth}%
		\setlength{\labelsep}{\biblabelsep}%
		\addtolength{\leftmargin}{\labelsep}%
		\setlength{\itemsep}{\bibitemsep}%
		\setlength{\parsep}{\bibparsep}}}
{\endlist}
{\item}

\newcommand{\lgpllicense}{%
	\begingroup\normalfont
	\includegraphics[height=2\fontcharht\font`\B]{lgpl-v3-logo}%
	\endgroup
}

\newcommand{\debian}{%
	\begingroup\normalfont
	\includegraphics[height=2\fontcharht\font`\B]{debian}%
	\endgroup
}

\newcommand{\ubuntu}{%
	\begingroup\normalfont
	\includegraphics[height=2\fontcharht\font`\B]{ubuntu}%
	\endgroup
}

\newcommand{\opensuse}{%
	\begingroup\normalfont
	\includegraphics[height=2\fontcharht\font`\B]{opensuse}%
	\endgroup
}

\newcommand{\archlinux}{%
	\begingroup\normalfont
	\includegraphics[height=2\fontcharht\font`\B]{archlinux}%
	\endgroup
}

\newcommand{\freebsd}{%
	\begingroup\normalfont
	\includegraphics[height=2\fontcharht\font`\B]{freebsd}%
	\endgroup
}

\newcommand{\MVAt}{{\usefont{U}{mvs}{m}{n}\symbol{`@}}}

\providecommand{\interval}[1]{
	\directlua{
		local oldsumstr,sumstr = difference(#1)
		tex.print("\string\\textbf{Time estimation}: " .. #1 .. " seconds." ..
		"\string\\hfill" .. "$t\string\\in" .. "\string\\left[" ..
				oldsumstr .. "," .. sumstr .. "\string\\right]" .. "$" .. " min.")
	}
}


\begin{document}

{
\usebackgroundtemplate{
	\centering
	\includegraphics[width=\paperwidth]{duneslides-logo}
}
\begin{frame}[plain,noframenumbering]

	\color{c++reviewduneblue}

	\begin{flushleft}\bfseries\scshape\huge
		Richards model for simulations of water infiltration in agricultural soil using DuMu\textsuperscript{x}
	\end{flushleft}

	\

	\

	\

	\

	\

	\

	\begin{minipage}{0.47\textwidth}
		\begin{figure}[ht!]
			\centering
			\includegraphics[height=2.5cm]{pec3}
			\caption*{
				\large
				\bfseries
				\textcolor{c++reviewduneblue}{
					Peruvian Conference on Scientific Computing
					October 4, 2022}
			}
		\end{figure}
	\end{minipage}
	\begin{minipage}{0.5\textwidth}
		\begin{flushright}
			\large
			\bfseries
			Made by\\
			John J. Leal Gomez\\
			Guillermo A. Martínez Girón\\
			Arley Martinez Jaramillo\\
			Universidad Nacional de Colombia\\
			Carlos Alonso Aznarán Laos\\
			Universidad Nacional de Ingeniería, Peru
		\end{flushright}
	\end{minipage}
	\note{
		En este webinar se presenta en dos partes, por un lado presentaremos el libro: \emph{Las matemáticas en la vida real, introducción básica al modelamiento matemático} y en segundo lugar se hará una presentación de la biblioteca modular DUNE.
	}
\end{frame}
}
% 1. Presentación del problema de infiltración de agua
% 1.1 Ventajas de la simulación % Arley ayuda, sin tantas ecuaciones.
% Optimizar el uso del agua en el suelo, lograr el mayor contacto entre las capas periféricas de la raíz del suelo. 1 gota de aceite contamina un litro de agua. riego por surco, problema del percolación, los nutrientes son llevados, ejemplo con el café tradicional. problema
% cantidad desmedida de agua porque el suelo pierde sus nutrientes, %60% de eficiencia
% ejemplo de la caña: pesticidas, exceso de agua, etc (ciclo).
% el perder agua es mucho más caro para la humanidad.
% aportar a la agricultura de precisión
% ruta apoplástica (explica la interacción de la raíz agua)
% % el efecto de las algas (dia/noche) se puede explicar con el cloropasto () y la mitocondria (respiración)
% qué solución hay.
% de lo general a particular.
% [1] Ref.
\section{Presentation of water infiltration problem}
% \subsection{Importancia}
% \subsection{Justificación}
% \subsection{Descripción del problema}
\subsection{Ecological footprint of the water cycle on agriculture}

\begin{frame}
	\frametitle{\subsecname}
	\begin{figure}[ht!]
		\centering
		\includegraphics[height=7cm]{contamination}
	\end{figure}
\end{frame}
% TODO: Poner la imagen en el centro, y en ovalos redondeados colocar las ideas del primer párrafo
% TODO: es decir, crecimiento poblacional, demanda de alimentos
% TODO: Agricultura que no es de precisión, agua.
% https://tex.stackexchange.com/questions/96289/how-to-connect-beamer-blocks-by-arrows
% \subsection{Problem}
\begin{frame}
	\frametitle{\secname}
	\begin{minipage}{0.6\textwidth}
		\begin{itemize}
			\item
			      Population and urban growth for years, has increased
			      both the demand for food and the exploitation of
			      resources such as \alert{water} and \alert{soil} for
			      agricultural activities.

			      However, in an effort to satisfy the demand, we have
			      neglected good management and stability of the surronunding
			      ecosystems~\cite{Latham1997}.
			      % El desarrollo urbano, en conjunto con el crecimiento poblacional
			      % a lo largo de los años, ha llevado al alza de la demanda de
			      % alimentos por parte del ser humano, motivo por el cual se se a
			      % visto en la obligación explotar cada vez más los recursos naturales que lo rodean y utilizar más terrenos para diversas actividades agrarias, sin embargo en su afán de suplir dicha demanda ha descuidado el buen manejo de los recursos naturales y la estabilidad de los ecosistemas que lo rodean.

			      \

			\item

			      One of the main problems with the greatest environmental
			      impact is the \alert{inefficient use of water management}
			      when implementing irrigation systems.

			      As a consequence,
			      large amounts of water are consumed, groundwater levels
			      are polluted, affecting the water footprint, and soils
			      deteriorate.

			      \textbf{Soil is a non-renewable resource!}
			      % Actualmente en la agricultura uno de los problemas más frecuentes  y de mayor impacto ambiental que genera el mal manejo de los cultivos, es el uso ineficiente de los recursos hídricos, al implementar sistemas de riego poco eficaces, al tratar satisfacer la demanda hídrica que generan las plantas. Además es uno de los puntos más importantes que se abordan en temas ambientales y que no solo preocupa por el desperdicio de agua sino también por el daño y las pérdidas que puede ocasionar en las propiedades fisicoquímicas del suelo.
			      % https://proain.com/blogs/notas-tecnicas/calidad-del-agua-para-riego-agricola
		\end{itemize}
	\end{minipage}
	\begin{minipage}{0.37\textwidth}
		\begin{figure}[ht!]
			\centering
			\includegraphics[height=4cm]{wasted_water}
		\end{figure}
	\end{minipage}
\end{frame}

\begin{frame}
	\frametitle{\secname}
	\begin{minipage}{0.5\textwidth}
		\begin{itemize}
			\item In Valle del Cauca, for example, one of the most
			      common crops is sugar cane, which generates contamination
			      problems for both the soil and groundwater (\alert{aquifers}).

			      \

			\item Currently, these aquifers supply more than
			      \alert{80\% of the useful water} used throughout the
			      region, and the impacts of fertilizers and inefficient
			      irrigation systems that currently exist are unknown.

		\end{itemize}
	\end{minipage}
	\begin{minipage}{0.47\textwidth}
		\begin{figure}[ht!]
			\centering
			\includegraphics[height=5.5cm]{water_waster2}
			% https://www.agromeat.com/284499/conceptos-tecnicas-y-estrategias-en-uso-eficiente-del-agua-de-riego
		\end{figure}
	\end{minipage}
\end{frame}
% Irricad
\section{Advantage of a Simulation}
\begin{frame}
	\frametitle{\secname}
	%\begin{minipage}{0.5\textwidth}
	\begin{itemize}
		\item The objective is to \alert{optimize the use of water} in
		      agricultural activities, including mathematical modelling.
		      The purpose is to study the \alert{Richard's equation} and
		      simulate the behavior in the water-soil interaction, trying
		      to deepen the understanding of the
		      \alert{infiltration phenomenon}.

		      \

		\item It also saves \alert{time} and \alert{money}, since several
		      variables and conditions of interest, such as temperature,
		      humidity, saturation, etc., can be included prior to field
		      testing.

		      \

		\item The simulation also allows studying various types of
		      soil and different interactions, for example with
		      \alert{organic matter}, \alert{plants}, \alert{nutrients}, etc.
		      These studies will allow the improvement of different plant
		      species and agricultural methods.

		      \

		\item It is also necessary to study how and for how long
		      contaminants infiltrate the soil and reach the aquifers.
	\end{itemize}
	%\end{minipage}
	%\begin{minipage}{0.47\textwidth}
	%	\begin{figure}[ht!]
	%		\centering
	%		\includegraphics[height=5.5cm]{water_waster2}
	% https://www.agromeat.com/284499/conceptos-tecnicas-y-estrategias-en-uso-eficiente-del-agua-de-riego
	%	\end{figure}
	%\end{minipage}
\end{frame}

\section{The soil and his features}
\begin{frame}
	\frametitle{\secname}
	\begin{minipage}{0.5\textwidth}
		\begin{itemize}
			\item Soils have \alert{different properties} like physical, mechanical, electrical, among others.

			      \

			\item They also have \alert{different pore sizes}, and thus are
			      classified, for which the \alert{textural triangle} is used.

			      \

			\item  Not all soils are the same, nor are they used for the same purposes,
			      so soil studies must be carried out to determine their \alert{characteristics}
			      and \alert{properties}.
		\end{itemize}
	\end{minipage}
	\begin{minipage}{0.47\textwidth}
		\begin{figure}[ht!]
			\centering
			\includegraphics[height=5cm]{propiedades-suelo}
		\end{figure}
	\end{minipage}
\end{frame}

\begin{frame}
	\frametitle{\secname}
	\begin{minipage}{0.47\textwidth}
		\begin{figure}[ht!]
			\centering
			\includegraphics[height=5.5cm]{textural_soil}
		\end{figure}
	\end{minipage}
	\begin{minipage}{0.5\textwidth}
		\lstinputlisting[
			caption={Spatial parameters \texttt{soil.params}.},
			label=soil.params,
		]{../../../script/soil.params}
	\end{minipage}
\end{frame}

\section{Soil water content} % Humedad en el suelo

\begin{frame}
	\frametitle{\secname}
	Depending on the porosity and characteristics, soils can retain
	water in different ways, the amount of water that cannot be easily
	obtained is called residual water, when there is a little more
	water, it is called \alert{permanent wilting} point because the
	plants cannot yet dispose of that water, it is strongly trapped by
	the pores.

	\

	When the soil has water available to the plant, but still has empty
	air spaces, it is called \alert{field capacity}.
	And finally, when it is completely filled with water, the soil is
	said to be \alert{saturated}.

\end{frame}
\begin{frame}
	\frametitle{\secname}
	\begin{figure}[ht!]
		\centering
		\includegraphics[height=6.8cm]{wetness}
	\end{figure}
\end{frame}
% https://i.pinimg.com/originals/34/66/77/34667762422298ffc10fec5252f28619.jpg

\section{Capillarity}

\begin{frame}
	\frametitle{\secname}
	\begin{center}\Huge % TODO: Poner sobre un fondo con relleno sin bordes.
		\fbox{What is the capillarity water soil?}
	\end{center}
\end{frame}
\begin{frame}
	\frametitle{\secname}
	Capillarity is a \alert{property of liquids} that depends on their
	surface tension, which, in turn, depends on the cohesion of the
	liquid, and which gives it the ability to move up or down a
	capillary tube.

	\

	When a liquid moves up a capillary tube, it is because the
	intermolecular force or \alert{intermolecular cohesion} is less than the
	adhesion of the liquid to the tube material; that is, it is a
	wetting liquid.
	The liquid continues to rise until the surface tension is balanced
	by the weight of the liquid filling the tube.

	\

	This is the case of \alert{water}, and it is this property that
	partially regulates its ascent within plants, without expending
	energy to overcome gravity.
\end{frame}

\begin{frame}
	\frametitle{\secname}
	\begin{minipage}{0.5\textwidth}
		\begin{align*}
			\Delta p=p_{\mathrm{a}}-p_{\mathrm{w}} & =
			\sigma_{\mathrm{aw}}\left(\frac{1}{r_{\mathrm{c} 1}}+\frac{1}{r_{\mathrm{c} 2}}\right)
			=\frac{2 \sigma_{\mathrm{aw}} \cos \psi}{r_{\mathrm{c}}}                                  \\[\baselineskip]
			\sigma_{\mathrm{aw}}\cos\psi           & =\sigma_{\mathrm{Sa}}-\sigma_{\mathrm{Sw}},\quad
			h_{\mathrm{c}} =\frac{1.5 \times 10^{-5}}{r_{\mathrm{c}}}
		\end{align*}
		\begin{itemize}
			\item $\sigma_{\mathrm{aw}}$ is the surface tension of the air-water interface.
			\item $\sigma_{\mathrm{Sa}}$ is the surface tension of the solid-air interface.
			\item $\sigma_{\mathrm{Sw}}$ is the surface tension of the solid-water interface.
			\item $\psi$ is the wetting angle.
		\end{itemize}
	\end{minipage}
	\begin{minipage}{0.47\textwidth}
		\begin{figure}[ht!]
			\centering
			\includegraphics[height=6.2cm]{capillar_preassure}
		\end{figure}
	\end{minipage}
\end{frame}


\section{Introduction}
% \subsection{DUNE for Multi-\{Phase, Component, Scale, Physics, \ldots\} flow and transport in porous media}

\begin{frame}
	\frametitle{\secname}
	% \framesubtitle{\subsecname}

	\begin{alertblock}{DUNE for Multi-\{Phase, Component, Scale, Physics, \ldots\} flow and transport in porous media (DuMu\textsuperscript{x})}
		\note{
			Título: Una introducción a la caja de herramientas DUNE en
			C++/Python para la solución de modelos matemáticos.

			Se hará una breve presentación de la caja de herramientas
			modular Dune Numerics, biblioteca modular desarrollada en la
			Universidad de Heildeberg en C++ y Python, para resolver
			ecuaciones diferenciales parciales utilizando métodos basados
			en mallas, por ejemplo diferencias finitas, elementos finitos o
			volúmenes finitos.

			Es un software de código abierto bajo la licencia GNU General
			Public Licence 2, con binarios disponibles para las
			distribuciones linux Debian, Ubuntu y openSUSE; y
			los scripts de compilación en macOS, FreeBSD, Arch Linux.

			Se mostrará la estructura general, algunos proyectos basados en
			DUNE y algunas simulaciones de modelos matemáticos que incluyen
			éste tipo de ecuaciones y sus respectivas soluciones, así como
			una implementación breve de Dune Numerics.
		}
		\begin{itemize}\small
			\item

			      DuMu\textsuperscript{x} is a multipurpose open-source simulator under the
			      \href{https://www.gnu.org/licenses/lgpl-3.0.html}{
				      GNU Lesser General Public License 3}~\lgpllicense{}.

			\item

			      DUNE is available on
			      \href{https://github.com/dune-copasi/homebrew-tap}{macOS},
			      \href{https://packages.debian.org/search?suite=sid&section=all&arch=any&searchon=sourcenames&keywords=dune-}{Debian}~\debian{},
			      \href{https://launchpad.net/~opm/+archive/ubuntu/ppa}{Ubuntu}~\ubuntu{},
			      \href{https://build.opensuse.org/search?search_text=dune-&search_for=2&name=1&attrib_type_id=}{openSUSE}~\opensuse{},
			      \href{https://aur.archlinux.org/packages/?O=0&SeB=n&K=dune-&outdated=&SB=n&SO=a&PP=50&do_Search=Ir}{Arch Linux}~\archlinux{}
			      and \href{https://www.freshports.org/search.php?stype=name&method=match&query=dune-&num=20&orderby=category&orderbyupdown=asc&search=Search&format=html&branch=head}{FreeBSD}~\freebsd{}.

			      {\fontspec[Renderer=Harfbuzz]{NotoColorEmoji.ttf}🎉}
			      DUNE Release 2.9.0 is planned for end of October 2022.

			\item

			      Porous-Medium Flow, Non-Isothermal, Free Flow, Geomechanics, Pore-Network models.
			      Multidomain, multi-component, multi-phase. Parallel, Grid Adaptivity.

				      {\fontspec[Renderer=Harfbuzz]{NotoColorEmoji.ttf}🎉}
			      Release 3.6.0 is planned for October 7, 2022.

			      \note{
				      Desarrollado con CMake, escrito en C++ con enlaces Python a través de pybind11.

			      }
		\end{itemize}
	\end{alertblock}

	\begin{minipage}{0.45\textwidth}
		\begin{figure}[ht!]
			\centering
			\includegraphics[height=3.2cm]{dunedesign}
			\caption{Taken from \url{https://dune-project.org}.}
		\end{figure}
	\end{minipage}
	\begin{minipage}{0.45\textwidth}
		\begin{figure}[ht!]
			\centering
			\href{https://repology.org/project/dumux/versions}{\includegraphics[height=0.7cm]{dumux-repology}} % dumux simulation picture
			\caption{.}
		\end{figure}
	\end{minipage}

\end{frame}

% documentación https://git.iws.uni-stuttgart.de/dumux-repositories/dumux-course
% página principal
% lenguajes
% https://player.vimeo.com/video/572717824
% https://cpp-review-dune.github.io/webinar/slides.pdf
% TODO: Mirar NON-ISOTHERMAL FLUID FLOW RESERVOIR SIMULATION USING DUMUX SOFTWARE
% \input{modules}
\section{Spatial parameters file}
\begin{frame}[fragile]
	\frametitle{\secname} % Properties, Spatial parameters
	\footnotesize
	\begin{minipage}{0.6\textwidth}
		\lstinputlisting[
			% caption={Programa \texttt{spatialparams.hh}.},
			label=spatialparams.hh,
			firstline=39,
			lastline=67
		]{../../richards/lens/spatialparams.hh}
	\end{minipage}
	\begin{minipage}{0.3\textwidth}
		\lstinputlisting[
			% caption={Programa \texttt{spatialparams.hh}.},
			label=spatialparams.hh,
			firstline=126,
			lastline=145
		]{../../richards/lens/spatialparams.hh}
	\end{minipage}
\end{frame}

\section{Results}
\begin{frame}
	\frametitle{\secname}
	\begin{minipage}{0.5\textwidth}
		\begin{figure}[ht!]
			\centering
			\includegraphics[width=7.2cm]{infiltration_theta_vs_deltaeta}
		\end{figure}
	\end{minipage}
	\begin{minipage}{0.3\textwidth}
		\begin{figure}[ht!]
			\centering
			\includegraphics[width=7.2cm]{richard_lens}
		\end{figure}
	\end{minipage}
\end{frame}
% \input{archiso}

\begin{frame}\transblindsvertical
	\frametitle{Referencias}
	%------------------------------------------------------------ 1
	\only<1>{
		\begin{itemize}
			\item

			      Books

			      \

			      \nocite{*}
			      \printbibliography[heading=none,keyword=book]
		\end{itemize}
	}

	\

	%------------------------------------------------------------ 2
	\only<2>{
		\begin{itemize}
			\item

			      Articles

			      \

			      \printbibliography[heading=none,keyword=paper]
		\end{itemize}
	}
\end{frame}

\begin{frame}\transblindsvertical
	\frametitle{Referencias}
	%------------------------------------------------------------ 3
	\begin{itemize}
		\item

		      Websites

		      \

		      \printbibliography[heading=none,keyword=online]
	\end{itemize}
\end{frame}
\begin{frame}
	\frametitle{Acknowledgment}
	\begin{center}\Huge
		Thank you so much!
	\end{center}
	\begin{figure}[ht!]
		\centering
		\href{https://www.pec3.org}{\includegraphics[height=1.7cm]{pec3}}\,
		\href{https://unal.edu.co}{\includegraphics[height=1.7cm]{unal}}\,
		\href{https://dune-project.org}{\includegraphics[height=1.5cm]{dune-logo}}\,
		\href{https://dumux.org}{\includegraphics[height=1.1cm]{dumux}}
	\end{figure}
	\vfill
	\begin{columns}
		\begin{column}{0.5\textwidth}
			\textcolor{c++reviewduneblue!50!c++reviewduneverde}{\href{https://laboratorios.unal.edu.co/geslab2021/servicios-lab-unal/labs-palmira}{\textsc{Laboratorio de Análisis Ambiental}}}

			Sede Palmira, Colombia.

			\

			\textcolor{c++reviewduneblue}{Slides available on:}
			\begin{center}
				\href{https://cpp-review-dune.github.io/webinar/slides.pdf}{\url{https://cpp-review-dune.github.io/webinar/slides.pdf}}
			\end{center}
			% \textcolor{c++reviewduneblue}{Grabación disponible en:}
			% \begin{center}
			% 	\href{https://player.vimeo.com/video/572717824}{\url{https://player.vimeo.com/video/572717824}}
			% \end{center}
		\end{column}
		\hfill
		\begin{column}{0.5\textwidth}
			\begin{flushright}
				Suggestions or questions to:

				\

				\href{mailto:jlealgom@unal.edu.co}{jlealgom\MVAt unal.edu.co}


				\href{mailto:gumartinezg@unal.edu.co}{gumartinezg\MVAt unal.edu.co}

				\href{mailto:amartinezj@unal.edu.co}{amartinezj\MVAt unal.edu.co}

				\href{mailto:caznaranl@uni.pe}{caznaranl\MVAt uni.pe}
			\end{flushright}
		\end{column}
	\end{columns}

\end{frame}

\end{document}

% 2. La ecuación de Richards
% 2.1. Deducción de la ecuación de Richards (1D o 2D) (Mencionar Darcy)
% 2.1.1 Solución analítica y sus curvas solución con gráficas.
% 2.2.1 Van Genutchen Brookes-Corey model (opcional, pero recomendable)
% 3. El suelo y las características
% 3.1 Parámetros del suelo (triángulo textural) Guillermo, variables físicas
% 3.2 Perfil del suelo, foto de laboratorio de suelos. % Ayuda de Arley, Guillermo.
% 4. Resultados de la simulación % Enlace a la página con el vídeo, un archivo tar.gz
% 4.1. DuMuX (1 hoja)
% 4.1.1 Modelos TFFA, black-box (posiblemente RichardsFOAM)
% 
% Bibliograf'ia

% https://player.vimeo.com/video/572717824
% https://cpp-review-dune.github.io/webinar/slides.pdf (spanish)