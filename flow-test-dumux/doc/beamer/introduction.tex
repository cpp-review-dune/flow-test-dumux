% 1. Presentación del problema de infiltración de agua
% 1.1 Ventajas de la simulación % Arley ayuda, sin tantas ecuaciones.
% Optimizar el uso del agua en el suelo, lograr el mayor contacto entre las capas periféricas de la raíz del suelo. 1 gota de aceite contamina un litro de agua. riego por surco, problema del percolación, los nutrientes son llevados, ejemplo con el café tradicional. problema
% cantidad desmedida de agua porque el suelo pierde sus nutrientes, %60% de eficiencia
% ejemplo de la caña: pesticidas, exceso de agua, etc (ciclo).
% el perder agua es mucho más caro para la humanidad.
% aportar a la agricultura de precisión
% ruta apoplástica (explica la interacción de la raíz agua)
% % el efecto de las algas (dia/noche) se puede explicar con el cloropasto () y la mitocondria (respiración)
% qué solución hay.
% de lo general a particular.
% [1] Ref.
\section{Presentation of water infiltration problem}
% \subsection{Importancia}
% \subsection{Justificación}
% \subsection{Descripción del problema}
\subsection{Ecological footprint of the water cycle on agriculture}

\begin{frame}
	\frametitle{\subsecname}
	\begin{figure}[ht!]
		\centering
		\includegraphics[height=7cm]{contamination}
	\end{figure}
\end{frame}
% TODO: Poner la imagen en el centro, y en ovalos redondeados colocar las ideas del primer párrafo
% TODO: es decir, crecimiento poblacional, demanda de alimentos
% TODO: Agricultura que no es de precisión, agua.
% https://tex.stackexchange.com/questions/96289/how-to-connect-beamer-blocks-by-arrows
% \subsection{Problem}
\begin{frame}
	\frametitle{\secname}
	\begin{minipage}{0.6\textwidth}
		\begin{itemize}
			\item
			      Population and urban growth for years, has increased
			      both the demand for food and the exploitation of
			      resources such as \alert{water} and \alert{soil} for
			      agricultural activities.

			      However, in an effort to satisfy the demand, we have
			      neglected good management and stability of the surronunding
			      ecosystems~\cite{Latham1997}.
			      % El desarrollo urbano, en conjunto con el crecimiento poblacional
			      % a lo largo de los años, ha llevado al alza de la demanda de
			      % alimentos por parte del ser humano, motivo por el cual se se a
			      % visto en la obligación explotar cada vez más los recursos naturales que lo rodean y utilizar más terrenos para diversas actividades agrarias, sin embargo en su afán de suplir dicha demanda ha descuidado el buen manejo de los recursos naturales y la estabilidad de los ecosistemas que lo rodean.

			      \

			\item

			      One of the main problems with the greatest environmental
			      impact is the \alert{inefficient use of water management}
			      when implementing irrigation systems.

			      As a consequence,
			      large amounts of water are consumed, groundwater levels
			      are polluted, affecting the water footprint, and soils
			      deteriorate.

			      \textbf{Soil is a non-renewable resource!}
			      % Actualmente en la agricultura uno de los problemas más frecuentes  y de mayor impacto ambiental que genera el mal manejo de los cultivos, es el uso ineficiente de los recursos hídricos, al implementar sistemas de riego poco eficaces, al tratar satisfacer la demanda hídrica que generan las plantas. Además es uno de los puntos más importantes que se abordan en temas ambientales y que no solo preocupa por el desperdicio de agua sino también por el daño y las pérdidas que puede ocasionar en las propiedades fisicoquímicas del suelo.
			      % https://proain.com/blogs/notas-tecnicas/calidad-del-agua-para-riego-agricola
		\end{itemize}
	\end{minipage}
	\begin{minipage}{0.37\textwidth}
		\begin{figure}[ht!]
			\centering
			\includegraphics[height=4cm]{wasted_water}
		\end{figure}
	\end{minipage}
\end{frame}

\begin{frame}
	\frametitle{\secname}
	\begin{minipage}{0.5\textwidth}
		\begin{itemize}
			\item In Valle del Cauca, for example, one of the most
			      common crops is sugar cane, which generates contamination
			      problems for both the soil and groundwater (\alert{aquifers}).

			      \

			\item Currently, these aquifers supply more than
			      \alert{80\% of the useful water} used throughout the
			      region, and the impacts of fertilizers and inefficient
			      irrigation systems that currently exist are unknown.

		\end{itemize}
	\end{minipage}
	\begin{minipage}{0.47\textwidth}
		\begin{figure}[ht!]
			\centering
			\includegraphics[height=5.5cm]{water_waster2}
			% https://www.agromeat.com/284499/conceptos-tecnicas-y-estrategias-en-uso-eficiente-del-agua-de-riego
		\end{figure}
	\end{minipage}
\end{frame}
% Irricad
\section{Advantage of a Simulation}
\begin{frame}
	\frametitle{\secname}
	%\begin{minipage}{0.5\textwidth}
	\begin{itemize}
		\item The objective is to \alert{optimize the use of water} in
		      agricultural activities, including mathematical modelling.
		      The purpose is to study the \alert{Richard's equation} and
		      simulate the behavior in the water-soil interaction, trying
		      to deepen the understanding of the
		      \alert{infiltration phenomenon}.

		      \

		\item It also saves \alert{time} and \alert{money}, since several
		      variables and conditions of interest, such as temperature,
		      humidity, saturation, etc., can be included prior to field
		      testing.

		      \

		\item The simulation also allows studying various types of
		      soil and different interactions, for example with
		      \alert{organic matter}, \alert{plants}, \alert{nutrients}, etc.
		      These studies will allow the improvement of different plant
		      species and agricultural methods.

		      \

		\item It is also necessary to study how and for how long
		      contaminants infiltrate the soil and reach the aquifers.
	\end{itemize}
	%\end{minipage}
	%\begin{minipage}{0.47\textwidth}
	%	\begin{figure}[ht!]
	%		\centering
	%		\includegraphics[height=5.5cm]{water_waster2}
	% https://www.agromeat.com/284499/conceptos-tecnicas-y-estrategias-en-uso-eficiente-del-agua-de-riego
	%	\end{figure}
	%\end{minipage}
\end{frame}

\section{The soil and his features}
\begin{frame}
	\frametitle{\secname}
	\begin{minipage}{0.5\textwidth}
		\begin{itemize}
			\item Soils have \alert{different properties} like physical, mechanical, electrical, among others.

			      \

			\item They also have \alert{different pore sizes}, and thus are
			      classified, for which the \alert{textural triangle} is used.

			      \

			\item  Not all soils are the same, nor are they used for the same purposes,
			      so soil studies must be carried out to determine their \alert{characteristics}
			      and \alert{properties}.
		\end{itemize}
	\end{minipage}
	\begin{minipage}{0.47\textwidth}
		\begin{figure}[ht!]
			\centering
			\includegraphics[height=5cm]{propiedades-suelo}
		\end{figure}
	\end{minipage}
\end{frame}

\begin{frame}
	\frametitle{\secname}
	\begin{minipage}{0.47\textwidth}
		\begin{figure}[ht!]
			\centering
			\includegraphics[height=5.5cm]{textural_soil}
		\end{figure}
	\end{minipage}
	\begin{minipage}{0.5\textwidth}
		\lstinputlisting[
			caption={Spatial parameters \texttt{soil.params}.},
			label=soil.params,
		]{../../../script/soil.params}
	\end{minipage}
\end{frame}

\section{Soil water content} % Humedad en el suelo

\begin{frame}
	\frametitle{\secname}
	Depending on the porosity and characteristics, soils can retain
	water in different ways, the amount of water that cannot be easily
	obtained is called residual water, when there is a little more
	water, it is called \alert{permanent wilting} point because the
	plants cannot yet dispose of that water, it is strongly trapped by
	the pores.

	\

	When the soil has water available to the plant, but still has empty
	air spaces, it is called \alert{field capacity}.
	And finally, when it is completely filled with water, the soil is
	said to be \alert{saturated}.

\end{frame}
\begin{frame}
	\frametitle{\secname}
	\begin{figure}[ht!]
		\centering
		\includegraphics[height=6.8cm]{wetness}
	\end{figure}
\end{frame}
% https://i.pinimg.com/originals/34/66/77/34667762422298ffc10fec5252f28619.jpg

\section{Capillarity}

\begin{frame}
	\frametitle{\secname}
	\begin{center}\Huge % TODO: Poner sobre un fondo con relleno sin bordes.
		\fbox{What is the capillarity water soil?}
	\end{center}
\end{frame}
\begin{frame}
	\frametitle{\secname}
	Capillarity is a \alert{property of liquids} that depends on their
	surface tension, which, in turn, depends on the cohesion of the
	liquid, and which gives it the ability to move up or down a
	capillary tube.

	\

	When a liquid moves up a capillary tube, it is because the
	intermolecular force or \alert{intermolecular cohesion} is less than the
	adhesion of the liquid to the tube material; that is, it is a
	wetting liquid.
	The liquid continues to rise until the surface tension is balanced
	by the weight of the liquid filling the tube.

	\

	This is the case of \alert{water}, and it is this property that
	partially regulates its ascent within plants, without expending
	energy to overcome gravity.
\end{frame}

\begin{frame}
	\frametitle{\secname}
	\begin{minipage}{0.5\textwidth}
		\begin{align*}
			\Delta p=p_{\mathrm{a}}-p_{\mathrm{w}} & =
			\sigma_{\mathrm{aw}}\left(\frac{1}{r_{\mathrm{c} 1}}+\frac{1}{r_{\mathrm{c} 2}}\right)
			=\frac{2 \sigma_{\mathrm{aw}} \cos \psi}{r_{\mathrm{c}}}                                  \\[\baselineskip]
			\sigma_{\mathrm{aw}}\cos\psi           & =\sigma_{\mathrm{Sa}}-\sigma_{\mathrm{Sw}},\quad
			h_{\mathrm{c}} =\frac{1.5 \times 10^{-5}}{r_{\mathrm{c}}}
		\end{align*}
		\begin{itemize}
			\item $\sigma_{\mathrm{aw}}$ is the surface tension of the air-water interface.
			\item $\sigma_{\mathrm{Sa}}$ is the surface tension of the solid-air interface.
			\item $\sigma_{\mathrm{Sw}}$ is the surface tension of the solid-water interface.
			\item $\psi$ is the wetting angle.
		\end{itemize}
	\end{minipage}
	\begin{minipage}{0.47\textwidth}
		\begin{figure}[ht!]
			\centering
			\includegraphics[height=6.2cm]{capillar_preassure}
		\end{figure}
	\end{minipage}
\end{frame}

