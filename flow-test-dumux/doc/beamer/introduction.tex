% 1. Presentación del problema de infiltración de agua
% 1.1 Ventajas de la simulación % Arley ayuda, sin tantas ecuaciones.
% Optimizar el uso del agua en el suelo, lograr el mayor contacto entre las capas periféricas de la raíz del suelo. 1 gota de aceite contamina un litro de agua. riego por surco, problema del percolación, los nutrientes son llevados, ejemplo con el café tradicional. problema
% cantidad desmedida de agua porque el suelo pierde sus nutrientes, %60% de eficiencia
% ejemplo de la caña: pesticidas, exceso de agua, etc (ciclo).
% el perder agua es mucho más caro para la humanidad.
% aportar a la agricultura de precisión
% ruta apoplástica (explica la interacción de la raíz agua)
% % el efecto de las algas (dia/noche) se puede explicar con el cloropasto () y la mitocondria (respiración)
% qué solución hay.
% de lo general a particular.
% [1] Ref.
\section{Presentation of water infiltration problem}
\subsection{Importancia}
\subsection{Justificación}
\subsection{Descripción del problema}
\subsection{Ecological footprint of the water cycle on agriculture}

\begin{frame}
	\frametitle{\subsecname}
	\begin{figure}[ht!]
		\centering
		\includegraphics[height=7cm]{contamination}
	\end{figure}
	\note{
		En la diapositiva se muestra una de estas situaciones diarias.

		También se tiene un anexo donde se estudia un software libre
		como es \lstinline|wxmaxima|.
	}
\end{frame}


\begin{frame}
	\frametitle{\secname}
	\begin{minipage}{0.6\textwidth}
		\begin{itemize}
			\item .
		\end{itemize}
	\end{minipage}
	\begin{minipage}{0.37\textwidth}
		\begin{figure}[ht!]
			\centering
			\includegraphics[height=4cm]{wasted_water}
			% https://proain.com/blogs/notas-tecnicas/calidad-del-agua-para-riego-agricola
		\end{figure}
	\end{minipage}
\end{frame}

\begin{frame}
	\frametitle{\secname}
	\begin{minipage}{0.5\textwidth}
		\begin{itemize}
			\item .
		\end{itemize}
	\end{minipage}
	\begin{minipage}{0.47\textwidth}
		\begin{figure}[ht!]
			\centering
			\includegraphics[height=5.5cm]{water_waster2}
			% https://www.agromeat.com/284499/conceptos-tecnicas-y-estrategias-en-uso-eficiente-del-agua-de-riego
		\end{figure}
	\end{minipage}
\end{frame}
% Irricad

\begin{frame}
	\begin{center}\Huge
		What is the capillarity water soil?
	\end{center}
\end{frame}

\subsection{Capillarity}

\begin{frame}
	\frametitle{\subsecname}
	\begin{minipage}{0.5\textwidth}
		\begin{align*}
			\Delta p=p_{\mathrm{a}}-p_{\mathrm{w}} & =
			\sigma_{\mathrm{aw}}\left(\frac{1}{r_{\mathrm{c} 1}}+\frac{1}{r_{\mathrm{c} 2}}\right)
			=\frac{2 \sigma_{\mathrm{aw}} \cos \psi}{r_{\mathrm{c}}}                                  \\[\baselineskip]
			\sigma_{\mathrm{aw}}\cos\psi           & =\sigma_{\mathrm{Sa}}-\sigma_{\mathrm{Sw}},\quad
			h_{\mathrm{c}} =\frac{1.5 \times 10^{-5}}{r_{\mathrm{c}}}
		\end{align*}
		\begin{itemize}
			\item $\sigma_{\mathrm{aw}}$ is the surface tension of the air-water interface
			\item $\sigma_{\mathrm{Sa}}$ is the surface tension of the solid-air interface
			\item $\sigma_{\mathrm{Sw}}$ is the surface tension of the solid-water interface
			\item $\psi$ is the wetting angle
		\end{itemize}
	\end{minipage}
	\begin{minipage}{0.47\textwidth}
		\begin{figure}[ht!]
			\centering
			\includegraphics[height=6.2cm]{capillar_preassure}
		\end{figure}
	\end{minipage}
\end{frame}

\subsection{Soil water content} % Humedad en el suelo
\begin{frame}
	\frametitle{\subsecname}
	\begin{figure}[ht!]
		\centering
		\includegraphics[height=6.8cm]{wetness}
	\end{figure}
\end{frame}